%导言区
\documentclass{ctexart}%book,report,letter
%\usepackage{ctex}
\usepackage{amsmath}
\usepackage{graphicx}
%\graphicspath{{QuestionI.png}}
\newcommand\degree{^\circ}
\newcommand{\myfont}{\textit{\textbf{\textsf{Fancy Text}}}}
\title{\kaishu 牛顿插值}
\author{Shirley}
\date{\today}

%正文区
\begin{document}
	三次Hermite插值
	
	\begin{equation}
		v_{ij}=\frac{max\left \{{x_{1j},x_{2j},\cdots ,x_{nj}}  \right \}-x_{ij} }{max\left \{{x_{1j},x_{2j},\cdots ,x_{nj}}  \right \}-min\left \{{x_{1j},x_{2j},\cdots ,x_{nj}}  \right \}} 
	\end{equation}
	\begin{equation}
		v_{ij}=\frac{x_{ij}-min\left \{{x_{1j},x_{2j},\cdots ,x_{nj}}  \right \}}{max\left \{{x_{1j},x_{2j},\cdots ,x_{nj}}  \right \}-min\left \{{x_{1j},x_{2j},\cdots ,x_{nj}}  \right \}} 
	\end{equation}
	\begin{equation}
		P_{ij}=\frac{v_{ij}}{\sum_{i=1}^{m}v_{ij} } (0\le v_{ij} \le 1,0\le p_{ij} \le 1)
	\end{equation}
	\begin{equation}
		\begin{aligned}
			e_{j}=-\frac{1}{\ln_{}{m}}\sum_{i=1}^{m}p_{ij}\ln_{}{(p_{ij})} 
		\end{aligned}
	\end{equation}
	\begin{equation}
		d_{j}=1-e_{j}   
	\end{equation}


	\begin{figure}[h]
		\centering
		\includegraphics[height=0.3\textheight]{QuestionI}
		\caption{QuestionⅠ}
	\end{figure}
	
\end{document}