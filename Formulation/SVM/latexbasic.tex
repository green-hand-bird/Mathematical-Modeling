%导言区
\documentclass{article}%book,report,letter
\usepackage{url}
\usepackage{ctex}
\usepackage{amsmath}
\usepackage{amssymb}
\usepackage{graphicx}
\usepackage{makecell} %实现表格内换行
\usepackage{underlin}
%\graphicspath{{QuestionI.png}}
\newcommand\degree{^\circ}
\newcommand{\myfont}{\textit{\textbf{\textsf{Fancy Text}}}}
\title{\kaishu 牛顿插值}
\author{Shirley}
\date{\today}

%正文区
\begin{document}
	决策边界:
	$$w^{T}X_{i}+b > 0,y_{i} > 0$$
	$$w^{T}X_{i}+b < 0,y_{i} < 0$$
	$$max margin(x,b)\qquad s.t.y_{i}(w^{T}+b) > 0$$
	$$margin(w,b)=min \, distance(w,b,x_{i})=min\frac{1}{\left | w \right | } \left | w^{T}x_{i}+b \right | $$
	损失函数:
	$$1-y_{i}(W^{T}x_{i}+b)\leqslant 0,loss=0$$
	$$1-y_{\bar{2} }(w^{T}x_{i}+b)>0,\quad loss=1-y_{i}(w^{T}x_{i}+b)$$
	问题可以表示为:
	$$min\sum_{i}^{N}(1-y_{i}(x^{T}x_{i}+b))_{+}+\lambda  \left \| w \right \| ^{2}$$
	
	令$\xi_{i}=1-y_{i}(w^{T}x_{i}+b),\xi_{i}\geqslant0$,分为两种情况:
	$$1\text{、}1-y_{i}(w^{T}x_{i}+b)>0 \Rightarrow \xi _{i}=1-y_{i}(w^{T}x_{i}+b) \Rightarrow y_{i}(wx+b)=1-\xi_{i} $$
	$$2\text{、}1-y_{i}(w^{T}x_{i}+b)\leqslant 0 \Rightarrow y_{i}(wx+b)\leqslant 1 \Rightarrow y_{i}(wx+b)\leqslant 1-\xi_{i}(\xi _{i}=0) $$
	综合上述两种情况,可直接写为$y_{i}(wx+b)\leqslant 1-\xi_{i}$
	则最优化函数可化为:
	$$min\frac{1}{2}w^{T}w+C\sum_{i=1}^{N}\xi_{i}  $$
	$$s.t.y_{i}(w^{T}x_{i}+b)\geqslant1-\xi_{i},\xi_{i}\geqslant0$$
	对偶函数等价为:
	$$\underset{min}{\alpha}\frac{1}{2} \sum_{i=1}^{N}\sum_{j=1}^{N}\alpha_{i}\alpha_{j}y_{i}y_{j}(x_{i}x_{j})-\sum_{i=1}^{n}\alpha_{i}$$
	$$s.t.\sum_{i=1}^{N}\alpha_{i}y_{i}=0$$
	$$0\leqslant \alpha_{i} \leqslant C,i=1,2,\cdots N$$
	对偶函数的解为:
	$$w=\sum_{i=1}^{N}\alpha_{i}y_{i}x_{i}$$
	$$b=y_{j}-\sum_{i=1}^{N}\alpha_{i}y_{i}(x_{i}\cdot x_{j})$$
	决策函数为:
	$$f(x)=\text{sign}(\sum_{1}^{N}\alpha_{i}y_{i}(x\cdot x_{i})+b^{*})$$
	KKT条件为:
	$$\frac{\mathrm{d}f}{\mathrm{d}x}=0,\frac{\mathrm{d}f}{\mathrm{d}b}=0,\frac{\mathrm{d}f}{\mathrm{d}\lambda }=0$$
	$$\lambda _{i}(1-y_{i}(w^{T}x_{i}+b))=0$$
	核函数可以对特征进行升维,由于高维空间运算量太大,因而使用低维的计算结果作为两个高维向量的内积:
	\begin{flalign*} %若去掉*,则公式后面有编号
		\phi (x_{1},x_{2})*\phi (x_{1}',x_{2}') & =(z_{1},z_{2},z_{3})*(z_{1}',z_{2}',z_{3}') \\
		&=(x_{1}^{2},\sqrt{2} x_{1}x_{2},x_{2}^{2})(x_{1}'^{2},\sqrt{2}x_{1}'x_{2}',x_{2}'^{2}) \\
		&=(x_{1}x_{1}'+x_{2}x_{2}')=(xx')^{2}=K(x,x')
	\end{flalign*}
	核函数等价于两个映射哈函数的内积,不过,这个映射函数不需要手动指出。因为当两个映射函数相乘时,内积的结果可以用核函数表示。而映射函数在最优化问题中都是成对出现的。即出现映射函数的地方都可以用核函数替代。
	如果用映射函数将x映射到高维空间,那么应该用高维向量替换x所在的位置:
	$$\frac{1}{2}\sum_{i=1}^{N}\sum_{j=1}^{N}\alpha_{i}\alpha_{j}y_{i}y_{j}(x_{i}x_{j})-\sum_{i=1}^{n}\alpha_{i} $$
	$$\frac{1}{2}\sum_{i=1}^{N}\sum_{j=1}^{N}\alpha_{i}\alpha_{j}y_{i}y_{j}(\phi(x_{i})\phi(x_{j}))-\sum_{i=1}^{n}\alpha_{i} $$
	$$\frac{1}{2}\sum_{i=1}^{N}\sum_{j=1}^{N}\alpha_{i}\alpha_{j}y_{i}y_{j}(K(x_{i}x_{j}))-\sum_{i=1}^{n}\alpha_{i} $$
	那么最终拟合的结果由高维向量表示为:
	$$f(x)=\text{sign}(\sum_{1}^{N}\alpha_{i}y_{i}(\phi(x)\phi(x_{i})+b^{*})$$
	$$f(x)=\text{sign}(\sum_{1}^{N}\alpha_{i}y_{i}(K(x,x_{i})+b^{*})$$
	高斯核函数(RBF)\\
	正太分布:
	$$f(x)=\frac{1}{\sqrt{2\pi}\delta} \text{exp}(-\frac{(x-\mu)^{2}}{2\delta^{2}} ) $$
	高斯分布:
	$$K(x,y)=e^{-\gamma \left \| x-y \right \|^{2} }$$
	
	$$L(w,b,\xi, \alpha, \mu)\equiv \frac{1}{2} \left \| w \right \| ^{2}+C\sum_{i=1}^{n}\xi _{i}-\sum_{i=1}^{n} \alpha _{i}(y_{i}(w\cdot x_{i}+b)-1+\xi _{i})-\sum_{i=1}^{n}\mu _{i}\xi _{i}
	$$
	$$w=\sum_{i=1}^{n}\alpha _{i}y_{i}\phi (x_{n}) $$
	


\end{document}